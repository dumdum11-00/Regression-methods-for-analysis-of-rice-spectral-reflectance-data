\section{Conclusion}

In this study, we work with the prediction of multiple nutrition type , chlorophyll content, Phosphorus concentration, and Potassium concentration. From the above comparison, we can conclude that the best result is the prediction of the chlorophyll content. This is because the dataset of Chlorophyll is pretty well which can be concluded that the wavelength when collecting data is good. The P and K concentration on the other hand is not as great as with Chlorophyll, especially the worst one among the fields is the K concentration because of its poor performance in MAPE and MSE. Luckily the  R2  score was not the greatest but it is still not so bad when compared to P concentration and Chlorophyll content. The possible causes for this bad performance in P and K may be because of the imbalance in measures P and K concentrations, or an insufficient amount of training data available for learning models. When comparing between applying or without utilising PCA, with 5 component can be a good choice to have better results. However, some of them show a decrease in prediction quality.
No models stand out too much to be use exclusively with all three nutrient concentration. To pick out the best model for each concentration: Chlorophyll predictions is best compatible with AdaBoost model; Bayesian Ridge model with or without PCA's result remain consistently best for P regression, and for K prediction, Random Forest model have the best calculation.
We also learned that PCA is not the only method to reduce the complexity of our dataset. Limiting bandwidth has proven that it can generate even better result: GradientBoosting models at 735-1084nm has the best result when it comes to predict Chlorophyll content; and ElasticNet has shown us that it excel others at both P, K prediction and even utilising very similar frequency: ~660-1426nm.
 
In the modern age, the integration of advanced technologies into agriculture has gained a great amount of popularity. Plant health is important in agriculture, and these technological advancements provide farmers with huge support. Farmers may optimize their operations by using the potential of these technologies, resulting in savings in both time and effort.
